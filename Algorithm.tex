\documentclass[journal]{IEEEtran}

\usepackage{algorithmic}
\usepackage{algorithm}
\usepackage{amssymb} % mathbb

\usepackage{xcolor}

\usepackage{mathtools} %floor
\DeclarePairedDelimiter\ceil{\lceil}{\rceil}
\DeclarePairedDelimiter\floor{\lfloor}{\rfloor}

% Redefine the rule colors
% WARNING: may have adverse effects on other float rule color (untested)
\makeatletter
\renewcommand\fs@ruled{\def\@fs@cfont{\bfseries}\let\@fs@capt\floatc@ruled
  \def\@fs@pre{{\color{red}\hrule height.8pt depth0pt \kern2pt}}%
  \def\@fs@post{{\color{red}\kern2pt\hrule\relax}}%
  \def\@fs@mid{{\kern2pt\color{red}\hrule\kern2pt}}%
  \let\@fs@iftopcapt\iftrue}
\makeatother

\definecolor{forestgreen}{rgb}{0.13, 0.55, 0.13}
\definecolor{green(html/cssgreen)}{rgb}{0.0, 0.5, 0.0}


\begin{document}

\title{A Compendium of Basic Algorithms}

\author{\IEEEauthorblockN{Melvin Cabatuan}\\
\IEEEauthorblockA{Electronics and Computer Engineering Department\\
De La Salle University\\
Manila, Philippines\\
Email: melvin.cabatuan@dlsu.edu.ph}}

% make the title area
\maketitle

\section{Introduction}
This lecture notes includes a collection of commonly
used algorithms in an introductory course for a computational background.
An algorithm is a finite set of precise instructions 
for performing a computation or for solving a problem.
An algorithm has input values from a specified set.
From each set of input values, an algorithm produces output values.
An algorithm should yield the correct output values for each set of input values.
Furthermore, the steps of an algorithm must be defined precisely and the desired
output must be produced after a finite number of steps.
Finally, the procedure should be applicable for all problems of the desired form, not just for a
particular set of input.

An algorithm is often described using a pseudocode which is a high-level description of an algorithm that uses the structural conventions of a standard programming language, but is intended for human reading. The following sections describe an algorithm in terms of a pseudocode with actual implementation.


\section{Finding the Maximum}
 

\floatname{algorithm}{\color{blue}Algorithm}
\begin{algorithm}
  \caption{\color{green(html/cssgreen)}Calculate $maximum$ element in an integer set}
  \label{alg1}
  \color{green(html/cssgreen)}
  \begin{algorithmic}
    \input{Pseudocode/FindMax.tex}
  \end{algorithmic}
\end{algorithm}


\section{Finding the Minimum}
 

\floatname{algorithm}{\color{blue}Algorithm}
\begin{algorithm}
  \caption{\color{green(html/cssgreen)}Calculate $minimum$ element in an integer set}
  \label{alg1}
  \color{green(html/cssgreen)}
  \begin{algorithmic}
    \input{Pseudocode/FindMin.tex}
  \end{algorithmic}
\end{algorithm}


\section{Linear Search (Using a For-Loop)}
 

\floatname{algorithm}{\color{blue}Algorithm}
\begin{algorithm}
  \caption{\color{green(html/cssgreen)}Locate an element $x$ in a list of distinct
values or determine that it is not in the list.}
  \label{alg1}
  \color{green(html/cssgreen)}
  \begin{algorithmic}
      \REQUIRE $\{a_1, a_2,\ldots, a_i, \ldots, a_n\}_{\ne} \in \mathbb{Z}$; $x \in \mathbb{Z}$
  \ENSURE $result = k $, where $ (a_k = x) $ and $ k \in \{1,\ldots,n\} $ if the element is found; otherwise $k = -1$ 

  \STATE $result \leftarrow -1$
  \FOR{$i=1$ to $n$}
      \IF{$result == a_i$}
      \STATE $result \leftarrow i$
      \ENDIF
  \ENDFOR

  \end{algorithmic}
\end{algorithm}
 
 
 
 \section{Linear Search (Using a While-Loop)}
 
 
 \floatname{algorithm}{\color{blue}Algorithm}
\begin{algorithm}
  \caption{\color{green(html/cssgreen)}Locate an element $x$ in a list of distinct
values or determine that it is not in the list.}
  \label{alg1}
  \color{green(html/cssgreen)}
  \begin{algorithmic}
    \input{Pseudocode/LinearSearchWhile.tex}
  \end{algorithmic}
\end{algorithm}


 \section{Binary Search}
 
 
 \floatname{algorithm}{\color{blue}Algorithm}
\begin{algorithm}
  \caption{\color{green(html/cssgreen)}Locate an element $x$ in a list of distinct
and sorted values or determine that it is not in the list.}
  \label{alg1}
  \color{green(html/cssgreen)}
  \begin{algorithmic}
      \REQUIRE $\{a_1, a_2,\ldots, a_i, \ldots, a_n\}_{\ne} \in \mathbb{Z}$, where $a_1<a_2<\ldots<a_n$; $x \in \mathbb{Z}$
  \ENSURE $result = k $, where $ (a_k = x) $ and $ k \in \{1,\ldots,n\} $ if the element is found; otherwise $k = -1$ 

  \STATE $i \leftarrow 1$
  \STATE $j \leftarrow n$
  
  \WHILE{$i < j$}
  	  \STATE $mid \leftarrow \floor*{\dfrac{i + j}{2}}$
  	  \IF{$x > a_{mid}$}
         \STATE $i \leftarrow mid + 1$
      \ELSE 
         \STATE $j \leftarrow mid$
      \ENDIF
  \ENDWHILE
  
  \IF{$x == a_i$}
      \STATE $result \leftarrow i$
  \ELSE 
      \STATE $result \leftarrow -1$
  \ENDIF

  \end{algorithmic}
\end{algorithm}
 

\end{document}
